\documentclass[12pt, a4paper]{article}
\title{Favole}
\author{Kristian Blido}
\date{03-03-2021}
\begin{document}
\maketitle
\section*{Volpe e cicogna}
La volpe e la cicogna sono diventate amiche. Un giorno la volpe invitò la sua amica a pranzo. La cicogna ha accettato volentieri l'invito. Non era ancora arrivato alla casa della volpe quando sentì il profumo dei piatti che lo aspettavano. La volpe lo ha accolto con tanto affetto. Era ora di mangiare. La volpe portò un grande vassoio e vi versò la meravigliosa zuppa che aveva cucinato.

"Eccolo, amico mio," disse la volpe.

La cicogna ha cercato di assaggiare qualcosa, ma invano. La volpe con la lingua ha lavato la deliziosa zuppa con un fiato. La cicogna taceva. Mentre si separavano, pregò la volpe di ricambiare la visita.

Anche la volpe poteva annusare il delizioso piatto da lontano. Cercò di indovinare cosa aveva preparato la sua amica per farli banchettare.

Anche la cicogna lo ha accolto con amore e gentilezza.

Era ora di mangiare. La cicogna portava un bulbo a bocca stretta dove poteva essere inserito solo il lungo becco di una cicogna.

La cicogna ha implorato la volpe di mangiare.

La volpe gli disse che aveva mangiato mentre guardava disperatamente la cicogna che non tirava fuori il becco dalla gola sottile del suo guscio.

Si rese conto di aver ricevuto in premio la stessa moneta che aveva dato qualche tempo prima.
\section*{Il leone e il topo}
C’era una volta, nella grande foresta, un maestoso leone, che si riposava all’ombra di un grande albero.
Stava controllando se in lontananza c’erano delle prede da poter cacciare, ma in quel momento non vedeva niente di interessante.
Così il pomeriggio passava lento. All’orizzonte non c’era nessuna preda da poter prendere e la pancia iniziava a brontolare dalla fame.

– Forse è meglio se mi sposto da qui e vado a cacciare in un’altra zona – si disse, abbastanza infastidito al pensiero di doversi alzare.
Ma proprio quando ormai aveva deciso di alzarsi ed andare via, ecco un piccolo topolino corrergli proprio davanti alle zampe.

Il leone colse al balzo l’occasione e, con uno scatto felino, bloccò la coda del topino con la zampa.
Il topino, che sperava di non essere visto, iniziò ad urlare disperato quando sentì di essere bloccato.
Il leone già pregustava il piccolo bocconcino come antipasto e si stava leccando i baffi.

Ma il topino, con le lacrime agli occhi iniziò a supplicarlo.
– Non mi mangiare, signor leone, ti prego non mi mangiare!
Il leone sorrise e iniziò a tirare con la zampa il topino verso di sé.
– Non mi mangiare, signor leone – continuò il topino – non ti sazierei che per pochi minuti da tanto sono piccolo.

Il leone pensò che questo era vero: quel topolino gli avrebbe placato la fame giusto per il tempo di alzarsi da lì.
– E poi le mie piccole ossicine rischierebbero di andarti di traverso in gola.
Anche questo era vero, pensò il leone, che smise di trascinare verso di sé il topolino.
– Se mi lascerai andare ti sarò riconoscente per tutta la vita! – disse infine il topo.
Il leone, mosso più dalla fatica di ingoiare quel piccolo pasto che dalla pietà per il topolino, lo lasciò andare.
– Vai topolino, forse un giorno ci rivedremo…
Il topolino ringraziò solennemente con grandi inchini e bacia-zampe, e poi scomparve tra le sterpaglie della foresta.

Il leone si decise infine ad andare in cerca di altre prede. Si incamminò dentro la foresta, ma dopo essere avanzato un po’ ecco che all’improvviso un legaccio fatto di corda lo intrappolò.
Il leone capì subito che quella era la trappola costruita da qualche cacciatore, e sapeva benissimo che da quel tipo di trappole non c’era scampo.

Il leone tirò con tutte le forze per cercare di liberarsi, ma più tirava, più il legaccio gli si stringeva alle zampe e gli faceva male. Dopo molti tentativi il leone si rassegnò, e si mise ad attendere il proprio destino.
Ma ad un tratto sentì qualcosa che stava lavorando sulla corda.
Guardò meglio e si accorse che il topolino di prima stava cercando di tagliare il legaccio con i suoi denti aguzzi.

– Non preoccuparti, signor leone, tra poco sarai di nuovo libero.
Il leone fu sorpreso dal gesto del topolino. Non si sarebbe mai aspettato che un animaletto così piccolo avrebbe potuto salvargli la vita.
– Topolino mio, io ti ho risparmiato la vita, e ora tu salvi la mia, questo ti fa grande onore!
Il topolino intanto lavorava veloce, e in pochi attimi il leone fu libero.

– Signor leone, quando si dà la parola d’onore, la si mantiene!
– Certo topolino mio e io ti ringrazio moltissimo per avermi liberato da questa trappola terribile. Ora siamo pari, e per tutta la vita anche io ti sarò riconoscente.

I due si salutarono, e andarono ognuno per la propria strada.
Ma il leone aveva imparato una lezione importantissima: bisogna essere gentili con tutti, anche con il più piccolo degli esseri viventi, perché l’aiuto più importante della vita potrebbe arrivare proprio da lì.

Morale: anche i più piccoli possono essere di grande aiuto, e chi è grande e forte non deve fare il prepotente.
\section*{La volpe e l’uva}
C’era una volta una volpe che vagava tranquilla per il bosco. Aveva appena bevuto ad un ruscello e si stava avventurando in cerca di cibo verso i campi coltivati, appena fuori dal paesello vicino.

Era già mattina inoltrata, e la fame iniziava a farsi sentire con sonori brontolii provenienti dal pancino.
Ad un certo punto, dopo aver camminato per un po’, vide una bella vigna piena di bellissimi grappoli d’uva.

La volpe controllò che non ci fossero pericoli in vista e si avvicinò furtiva ad uno dei grappoli, quello che le sembrava più vicino.
Non c’era nessuno nelle vicinanze. Era il momento perfetto per fare un bel salto e prendersi il grappolo d’uva!

La volpe quindi prese la rincorsa e… hop! Fece un balzo cercando di afferrare coi denti il grappolo, ma niente: non ci arrivò.
La volpe allora prese un po’ più di rincorsa e hop! Fece un altro balzo, ma anche questo non era abbastanza alto per riuscire ad arrivare al grappolo d’uva.
La volpe allora provò a prendere una rincorsa ancora più lunga e hop! Niente, non arrivò a prendere il grappolo d’uva.
Intanto il suo pancino brontolava sempre più dalla fame.

La volpe provò e riprovò. Le mancava sempre un soffio per prendere il grappolo d’uva ma non c’era verso, non riusciva ad arrivarci.
Stremata dalla fatica e dalla fame, la povera volpe guardò se nella vigna c’erano altri grappoli, magari più bassi, da poter prendere. Ma niente, erano tutti più in alto di quel grappolo che lei aveva cercato con tutte le sue forze di acciuffare.

La volpe diede un ultimo lungo sguardo al bel grappolo d’uva che tanto aveva sognato di mangiare, e per non ammettere di non essere riuscita nella sua impresa, si disse:
– Meglio così, tanto di sicuro quel grappolo era ancora acerbo e mangiarlo mi avrebbe solo fatto venire mal di pancia! – anche se sapeva benissimo che non era vero.

Così, sconsolata e ancora più affamata, ritornò con la coda tra le gambe nel suo boschetto. Si mise a caccia di qualcos’altro da mangiare, cercando questa volta di adocchiare qualcosa che avrebbe sicuramente preso.

Morale: è facile e tipico di chi è arrogante disprezzare ciò che non può avere. Meglio impegnarsi con umiltà per ottenerlo.
\end{document}
