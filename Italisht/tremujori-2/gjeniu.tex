\documentclass[12pt, a4]{article}
\title{La Vita di Leonardo da Vinci}
\author{Kristian Blido}
\date{3, Marzo 2021}
\begin{document}
\maketitle


    Leonardo da Vinci nacque nel villaggio di Vinci in Italia nel 1452 durante il Rinascimento, un periodo di rinascita di cultura e di arti. Era il figlio illegittimo di Ser Piero da Vinci, un notaio fiorentino, ed una contadina di nome Caterina.

    Da Vinci, già da età giovanissima dimostrò un talento speciale. All'età di otto anni, divenne apprendista di un pittore. Suo padre lo portò a Firenze dove fu addestrato nella pittura e nella scultura fiorentina. Imparò i rudimenti delle arti plastiche del tempo nell' officina del Verrocchiodove studiò per molti anni. Finalmente nel 1478 stabilì il suo proprio studio (bottega) in Firenze.

    In Firenze, il senso di concorrenza artistica era profondo fra gli artisti durante la fine del quindicesimo secolo. Dapprima, l'abilità di Leonardo non fu veramente apprezzata. Fu in quel periodo che Leonardo decise di dedicarsi anche allo studio della matematica e della scienza naturale. Prima di traslocarsi nel 1481, i Monaci di San Donato a Scopeto lo incaricarono di di dipingere un affresco per l'altare della Cappella. Leonardo così creo` "l'Adorazione del Magi."

    Nella sua decisione di abbandonare l'arte, Leonardo scrisse una lettera al Duca di Milano e si raccomandò come ingegnere militare e civile. Alle fine della sua lettera disse delle sue abilita` come architetto, scultore e pittore. Abbandonò Firenze nel 1482 ed entrò al servizio del Duca. La Corte di Milano l'incoraggiò a sviluppare i suoi talenti in qualsiasi campo artistico e scientifico in cui fosse interessato.

    Da 1492 a 1499 per ordine del Duca di Milano Leonardo fu impegnato in varie imprese nel campo artistico, edilizio ed anche militare. Come architetto fece parte di un concorso con lo scopo di disegnare la cupola del Cattedrale Milano e di un progetto per un palazzo per un nobile Milanese. Purtroppo nè l'uno nè l'altro progetto fu portato a termine. Come direttore coreografico della corte degli Sforza di Milano disegnò costumi e macchinari teatrali con apparati di palcoscenico per la produzione dei drammi. Come scultore passò anni a creare e a modellare una statua monumentale di Francesco Sforza, anche se non fu mai completata.

    Uno dei suoi primi progetti completati fu un affresco per l'altare per la Confraternità della Concezione Immacolata. La chiamò "La Vergine delle Rocce."

    Tra il 1495 e il 1497 dipinse "il Cenacolo" in Santa Maria delle Grazie. Quest'affresco fu dipinto in una vernice della tempera in una mistura diversa che Leonardo non aveva mai usato prima. Leonardo adoperò questa nuova combinazione di tempere nella speranza che offrisse il dettaglio e la ricchezza di una pittura ad olio su tela. Purtroppo, poco tempo dopo aver dipinto quest'affresco, l'opera cominciò a deteriorare, e a sbiadirsi quasi immediatamente.

    Nel 1499 i francesi invasero Milano e Leonardo fu costretto ad abbandonare la città. Ritornò a Firenze nel 1500 e durante questo periodo dipinse "La Vergine e Figlio con Sant'Anna". Quest'affresco mostrava Anna seduta sul grembo di sua madre. Anna provava a tenere il bambino Gesù Cristo lontano da un piccolo agnello. Non riuscì a completare questa pittura, in particolare il vestito di Maria.

    Nel 1502 Leonardo entrò al servizio di Cesare Borgia, il condottiere papale, come ingegnere militare. Restò alla corte del Borgia per dieci mesi e fece molti viaggi nelle zone centrali d'Italia. Creo` molte mappe che sono ora sono considerati monumenti importanti nella storia della cartografia.

    Durante questo tempo dipinse la Mona Lisa (la Gioconda). Uno dei cittadini più nobili di Firenze, Francesco di Bartolommeo di Zanobi del Giocondo, domandò a Leonardo di dipingere un ritratto della sua terza moglie, Lisa di Antonio Maria di Noldo Gherardini, o Mona Lisa. Comminciò il dipinto nel 1503 e impiegò quattro anni a finirla. Mona Lisa aveva ventiquattro anni quando Leonardo cominciò il dipinto. Dopo aver completato "la Gioconda" Leonardo decise di non consegnare il dipinto a Francesco, ma lo vendette al Re Francesco I di Francia.

   Leonardo si trasferì di nuovo a Milano dove visse dal 1506 al 1513. Ritornò una volta a Firenze nel 1507. Dalla fine del 1513 lo troviamo a Roma dove passò i prossimi due anni. Durante questo tempo non dipinse molto. Tenne quaderni del suo lavoro che contenevano gli studi anatomici, matematici e meccanici che Leonardo aveva fatto.

    Nel 1517 Leonardo ritornò in Francia dove fu proclamato pittore, architetto ed ingegnere della corte. Il due maggio, 1519, Leonardo da Vinci morì in Cloux, una città in Francia circondato da una corte di ammiratori.

    Molta gente ammirò Leonardo per la sua arguzia, la sua bellezza, le sue creazioni, ed il suo intelletto. Aveva un grande amore per la natura. A volte comprava degli uccelli in gabbia solo per il piacere di liberarli.

    Leonardo da Vinci scrisse con la mano sinistra e scriveva così dalla destra alla sinistra ed indietro. Questo gli sembrò un movimento "piu` naturale". Comunque, un altra importante ragione per cui scrisse in questo modo fu per impedire ad altri di leggere e analizzare il suo lavoro e le sue idee.

    Soprattutto, Leonardo da Vinci fu un "Uomo Rinascimentale." Dimostrò talento in tutti i campi. Il suo genio illimitato gli permise di eccellere in tanti campi diversi come la pittura, la scultura, la matematica e l'anatomia umana.

    Penso che Leonardo da Vinci sia stato dotato di molto talento. Le sue scoperte e le sue invenzioni sono numerose particolarmente in riguardo all'osservazione e allo studio della natura e del corpo umano. Gli affreschi che dipinse sono bellissimi. Preferisco "L Adorazione del Magi" di più perchè dimostra la bellezza delle donne in questo secolo. Da Vinci fu quasi perfetto come uomo. Fu educato in tutti campi e poteva fare molte cose. Ha inventato il paracadute, la migliatrice e macchine per la guerra. Leonardo da Vinci rappresenterà per sempre il vero simbolo dell' "uomo Rinascimentale." 
\end{document}
