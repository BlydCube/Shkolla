\documentclass[12pt, a4paper]{article}

\usepackage[utf8]{inputenc}
\usepackage[T1]{fontenc}
\usepackage{textcomp}
\pdfsuppresswarningpagegroup=1

\title{Book Review: \\ Origin by Dan Brown}
\author{Kristian Blido}
\date{19, February 2021}
\begin{document}
	\maketitle
Can a Harvard professor hide from a ruthless assassin? Can a bishop be homosexual? Can a computer stand above god? "Origin" is the latest novel by Dan Brown in the professor's Langdon saga. It is a story not only full of action also full of interesting real world facts.

Robert Langdon, a professor of symbology and religious iconology, attends Edmond Kirsch's discovery unveiling. This discovery is advertised to be so impactfull it will change the face of science, such breakthrough will challenge the fundamentals of human existence. The religious leaders do not approve of this event at all, however Edmond ignores them. During the ceremony before the big reveal, Edmond is killed and with him, the discovery as well. Robert, a guest as well as an Edmond's friend, grabs his phone and flees away while simultaneously trying to finish what Kirsch couldn't. 

The characters of the book are all well balanced and compelling in their own ways. Robert is more often than not calm and logical. Edmond, although dead for the most part, is shown as a very intelligent, thoughtful and always ready to adopt new ideas also a very vocal atheist. Ambra, who accompanied Robert, is the new princess of Spain and is already sick of it. Winston is an artificial intelligence system developed by Kirsch and helped Langdon.

The story all takes palce in Spain, from the Guggenheim museum in Bilbao to Sagrada Família and Casa Mila in Barcelona. The locations are very realistically described which makes the book a source of information while still being a source of entertainment. 

All in all "Origin" is one of the best novels I've ever read and I would recommend to whoever appreciates some easy to read, realistic stories. I challenge you to not like this book. 

\end{document}
