\documentclass[12pt, a4paper]{article}
\title{Anketimi}
\author{Kristian Blido}
\date{04-03-2021}
\begin{document}
	\maketitle
	\section*{Anketimi si metode kerkimi sasior}
	Anketimi është metoda bazë e kërkimit sasior. Ajo është një nga metodat më të hershme të studimit të dukurive të jetës shoqërore. Sikurse vëzhgimi, intervistimi, eksperimenti etj., edhe anketimi përdoret në shumicën e shkencave sociale.

	\section*{Llojet e anketimit}
	\subsection*{Shtrirja}
Për nga shtrirja ai mund të jetë i përgjithshëm dhe i pjesshëm. Një anketim i përgjithshëm është, për shembull, ai që kryhet për regjistrimin e popullsisë dhe të banesave, ngaqë shtrihet në të gjitha familjet e një shteti të dhënë. Por anketimi më i zakonshëm është anketimi i pjesshëm, anketimi me kampionim.

\subsection*{Lloji i pyetjeve}
Për nga lloji i pyetjeve, anketimi mund të bëhet me pyetje të hapura dhe me pyetje të mbyllura. Në rastin e parë i anketuara ka më tepër liri për dhënien e përgjigjes së tij. Në rastin e anketave me pyetje të mbyllura, i anketuari do të përgjigjet vetëm duke përzgjedhur një apo disa nga alternativat e parashtruara nga hartuesi i anketës.

\subsection*{Menyra e plotesimit}
Mënyra e plotësimit të anketës, është gjithashtu, një kriter për vlerësimin e anketimit. Anketa mund të plotësohet individualisht nga i anketuari. Por anketat mund të plotësohen edhe në forma të tjera. Një ndër to është në salla të veçanta dhe e plotësojnë anketën njëkohësisht. Anketa mund të plotësohet nga anketuesi edhe me përgjigje të marra përmes telefonit, internetit etj.

\subsection*{Identiteti}
Anketimi  mund të jetë anonim dhe apo i hapur. Në rastin e parë informacioni jepet në mirëbesim dhe identiteti i të anketuarit nuk bëhet i ditur. Por, në vartësi të problemit të studiuar, anketimi mund të jetë edhe i hapur, plotësisht transparent.

\subsection*{Perseritja}
Anketimi mund të jetë i veçantë (unik), ose anketim që përsëritet. Në rastin e parë nuk parashikohet paraprakisht përsëritja dhe ballafaqimi i rezultateve të një anketimi tip. Por ai mund të përdëritet me një periodicitet të fiksuar. Për shembull, regjistrimi i popullsisë bëhet, si rregull, një herë në dhjetë vjet.

\subsection*{Kodimi}
Për nga niveli i kodimit anketa mund të jetë plotësisht ose pjesërisht e koduar. Kodimi përfaqëson ekuivalentimin e çdo përgjigje me një numër, apo simbol. Anketimi i pakoduar është anketimi me pyetje të hapura.
\section*{Anketimi dhe Anketa}
Anketimi, si metodë e grumbullimit të të dhënave kërkimore, ka një “instrumet” bazë, që është anketa. 


Anketa përfaqëson një pyetësor, një dokument të shkruar në të cilin renditen disa pyetje. Pyetjet e anketës u drejtohen një numri të dhënë individësh. Individët përgjigjedhënës përzgjidhen sipas një procedure rigorozisht të caktuar. Përgjigjet e dhëna nga individët e anketuar përbëjnë informacionin e anketimit. 


Pyetjet e anketimit, si rregull, janë të koduara. Me anë të kodimit informacioni kërkimor i marrë përmes anketimit shndërrohet në numra. Informacioni numerik hidhet në kompjuter. Programet kompjuterike mundësojnë kuahtifikimin e opinioneve të dhëna nga individët e anketuar, pra shndërrimin e informacionit në numra. Për këtë arsye, anketimi konsiderohet metodë sasiore kërkimi.

Avantazhi kryesor i anketimit lidhet me faktin se nëpërjmet tij bëhet i mundur:
\begin{enumerate}
\item Mbledhja e një informacioni shumë të gjerë; 
\item Për një masë të madhe njerëzish; 
\item Brenda një kohe të shkurtër.
\end{enumerate}

Anketimi bëhet, si rregull, për probleme e dukuri që syri i vëzhguesit nuk mund t’i kapë dhe as intervistimi mund të mos i zbulojë.

\end{document}
