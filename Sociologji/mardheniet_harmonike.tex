\documentclass[a4paper]{article}

\usepackage[utf8]{inputenc}
\usepackage[T1]{fontenc}
\usepackage{textcomp}
\usepackage{amsmath, amssymb}

\title{\textbf{Detyre Permbledhese:} \\ Mardheniet harmonike ndermjet shqipetareve dhe pakicave etnike dhe kombetare.}
\author{Kristian Blido}
\date{24-02-2021}
\begin{document}
	\maketitle
	\section*{Ligji}
	\subsection*{Perkufizimi ligjor}
\begin{enumerate}
	\item Pakicë  kombëtare  është  një  grup  shtetasish  shqiptarë  që  banojnë në  territorin  e  Republikës  së  Shqipërisë,  kanë  lidhje  të  hershme  dhe  të  qëndrueshme  me  shtetin  shqiptar, shfaqin karakteristika dalluese kulturore, etnike, gjuhësore, fetare ose tradicionale dhe të cilët janë  të  gatshëm  për  të  shprehur,  ruajtur  dhe  zhvilluar  së  bashku  identitetin  e  tyre  të dallueshëm kulturor, etnik, gjuhësor, fetar ose tradicional.
	\item Në  kuptimin  e  këtij  ligji, pakicat  kombëtare  në  Republikën  e  Shqipërisë  janë  pakicat  greke, maqedonase, arumune, rome, egjiptiane, malazeze, boshnjake, serbe dhe bullgare.
\end{enumerate}
\subsection*{Ushtrimi i te drejtave}
\begin{enumerate}
	\item Çdo  person, që  i  përket  një  pakice  kombëtare,   ka  të  drejtën  të  zgjedhë  lirisht  që  të  trajtohet ose jo si i tillë,  duke mos pasur asnjë disavantazh nga kjo zgjedhje  ose nga ushtrimi i të drejtave që janë të lidhura me këtë zgjedhje.
	\item Personat,  që u  përkasin  pakicave  kombëtare,  i  ushtrojnë  të  drejtat  dhe gëzojnë liritë  e  garantuara  me  këtë  ligj,  në  mënyrë  individuale  dhe  në  komunitet  me  të  tjerët,   në  të  gjithë  territorin e Republikës së Shqipërisë.
\end{enumerate}
\subsection*{Ndalimi i diskriminimit}
\begin{enumerate}
	\item  Ndalohet  çdo  diskriminim  ndaj  cilido  personi  për  shkak  të  përkatësisë  së  tij/së saj  në një pakicë kombëtare. 
	\item Institucionet   publike,   qendrore   dhe   vendore,    miratojnë dhe   zbatojnë   masat   e   nevojshme:
	\begin{enumerate}
		\item për të garantuar barazi të plotë dhe efektive në jetën ekonomike, shoqërore, politike dhe kulturore  ndërmjet  personave  që  i  përkasin  një  pakice  kombëtare  dhe  atyre  që  i  përkasin  shumicës;
		\item për   të   mbrojtur   personat   që   u   përkasin   pakicave   kombëtare   nga   kërcënimet,   diskriminimi, armiqësia  apo  dhuna  për  shkak  të  identitetit  të  tyre  të  dallueshëm  kulturor, etnik, gjuhësor, fetar ose tradicional;
		\item për të forcuar dialogun ndërkulturor;
		\item për të nxitur respektin e ndërsjellë, mirëkuptimin dhe bashkëpunimin ndërmjet të gjithë qytetarëve  të  Republikës  së  Shqipërisë,  pa  dallim  për  sa  i  përket identitetit  të  tyre  të  dallueshëm kulturor, etnik, gjuhësor, fetar ose tradicional.
	\end{enumerate}
\item Masat  e  miratuara  në  përputhje  me  pikën 2,    të  këtij  neni, nuk  përbëjnë  akte  diskriminimi.
\end{enumerate}
\section*{Jo Ligj}
1930, Dhoma e Përhershme e Drejtësisë Ndërkombëtare thote se: "Ekzistenca e një minoriteti është një  çështje fakti dhe jo një çështje e së drejtës, jo një çështje thjesht juridike". 

Në këtë kuptim, ekzistenca e minoriteteve në Shqipëri është një realitet  sa historik aq edhe aktual, të cilit i është kushtuar një vëmendje e veçantë për  të konkretizuar një marrëdhënie të mirë, ku shprehet toleranca, bashkëjetesa  dhe mirëkuptimi mes pjestarëve të pakicave përkatëse dhe pjesës tjetër të  popullatës. Ky kujdes i kultivuar në mentalitetin e shoqërisë shqiptare, me  elemente diversiteti si pjesë të rëndësishme të trashëgimisë kulturore, ka  reflektuar një bashkëjetesë harmonike dhe tolerante, pa prezencën e  konflikteve etnike, racore, apo fetare. 

Me vendosjen e demokracisë në vend, trajtimi i pakicave ka marrë një  dimension të ri, fakt që duket qartë në angazhimet që shteti shqiptar ka  ndërmarrë për këtë qëllim. 

Sot në Shqipëri njihen dy lloje minoritetesh, minoritetet etnike  kombëtare ku përfshihet minoriteti Grek, minoriteti Maqedon dhe minoriteti  Serbo-Malazez, si dhe minoritetet etno gjuhësore, ku përfshihen minoriteti  Vllah dhe minoriteti Rom. Pavarësisht kësaj realiteti i shoqërisë sonë njeh “de  facto” edhe disa kominutete të tjera të ndryshme nga popullsia etnike shqiptare  siç janë komuniteti Boshnjak, komuniteti Egjiptian, apo edhe komuniteti Goran.  

Në fakt Kushtetuta e Republikës së Shqipërisë në nenin 20, shprehet  vetëm për personat që u përkasin pakicave kombëtare gjë që krijon edhe një  debat në lidhje me njohjen e zgjeruar nga ana e shtetit edhe të kategorisë së  minoriteteve etno gjuhësore.

I gjithë diskutimi në këtë aspekt përqëndrohet në dy drejtime, së pari,  nocioni juridik i përcaktimit të minoritetit dhe së dyti, vullneti shtetëror i  shprehur formalisht në nohjen zyrtare të komuniteteve të caktuara që jetojnë  në territorin e Republikës së Shqipërise dhe që efektivisht janë shtetas  shqiptarë, por që kanë elementë të jetesës dhe trashëgimisë të tyre të  ndryshme nga popullsia etnike shqiptare. 

Sa i takon nocionit juridik të përcaktimit të një përkufizimi mbi minoritetet  mund të thuhet se, ekzistenca e një shumice tekstesh ndërkombëtare, por  edhe e një sërë aktesh ndërkombëtare që trajtojnë këtë fushë, nuk japin ndonjë  përkufizim të përgjithshëm të nocionit të minoritetit, ku mund të përfshiheshin  të gjitha grupet e ndryshme minoritare. Përvoja ndërkombëtare ka treguar nga  ana tjetër se është e vështirë të jepet një përkufizim, ku mund të përfshihen të  gjitha kategoritë e minoriteteve, për shkak të larmisë së madhe të grupeve  minoritare dhe të vështirësive objektive për t’i klasifikuar ato në mënyrë  homogjene. Kjo gjë ka ndikuar edhe në mungesën e një përkufizimi të  përgjithshëm mbi minoritetet në aktet ndërkombëtare.

Megjithatë mungesa e këtij përkufizimi nuk ka penguar shtetet e  ndryshme, sikurse edhe Shqipërinë të njohin kategori të caktuara minoritetesh  bazuar në tregues të natyrës objektive dhe subjektive të komuniteteve të  caktuara në vend. Elementët objektivë lidhen me ekzistencën, brenda një  shteti të caktuar, të grupeve të veçanta të popullsisë që kanë karakteristika  etnike, fetare e gjuhësore të qëndrueshme. Kurse elementët subjektivë lidhen  me vullnetin për të ruajtur karakterin e veçantë të grupit, fakt që shman asimilimin. Mbi këtë bazë vlerësimi, e cila referohet në një protokoll të  Asamblesë Parlamentare të Këshillit të Europës, të vitit 1993, ku bëhet një  tentativë për dhënien e një përkufizimi mbi minoritetet,1 përcaktohet edhe fakti  se, grupet që i pohojnë dallimet e tyre gëzojnë një trajtim special, ndërsa grupet  e asimiluara mbeten jashtë këtij trajtimi. 

Njohja e minoriteteve nxjerr nevojën edhe të mbrojtjes së tyre, përmes  afirmimit të një sërë të drejtave që u njihen posaçërisht atyre. Sa i takon realitetit  tek ne mbi këto të drejta, Kushtetuta e Republikës së Shqipërisë, ka  sanksionuar në nenin 3 të saj, pluralizimin, identitetin kombëtar dhe  trashëgiminë  ka  sanksionuar  parimin  e  barazisë  para  ligjit  dhe  mosdiskriminimit, pavarësisht përkatësisë së individit në pakicën përkatëse.  Për më tepër, në nenin 20 të saj, në mënyrë eksplicite parashikohet që  personat të cilët u përkasin pakicave kombëtare ushtrojnë në barazi të plotë  para ligjit të drejtat dhe liritë e tyre. Ata kanë të drejtë të shprehin lirisht, pa u  ndaluar, as detyruar përkatësinë e tyre etnike, kulturore, fetare e gjuhësore,  si dhe kanë të drejtë t’i ruajnë e zhvillojnë ato. Kushtetuta iu ka njohur edhe të  drejtën të mësojnë dhe të mësohen në gjuhën e tyre amtare, si dhe të  bashkohen në organizata e shoqata për mbrojtjen e interesave dhe të identitetit  të tyre.

Anëtarësimi i Shqipërisë në një sërë organizatash dhe organizmash  ndërkombëtarë me fushë veprimtarie mbrojtjen e të drejtave të njeriut, si dhe  ratifikimi, apo nënshkrimi i një numri të konsiderueshëm aktesh ndërkombëtare  të qëllimuara në afirmimin dhe mbrojtjen e të drejtave të njeriut, veçanërisht në  referencë të minoriteteve, ratifikimi pa asnjë rezerve i Konventës Kuadër “Për  Mbrojtjen e Minoriteteve Kombëtare”, flasin për një angazhim serioz  institucional. Që prej 13 Korrikut të vitit 1995, Shqipëria është vend anëtar i  Këshillit të Europës, gjë që flet për këtë angazhim.

Zhvillime pozitive janë evidentuar edhe në kuadër të plotësimit të  kërkesave të Marrëveshjes së Stabilizim-Asociimit dhe të gjithë procesit në  vazhdim për anëtarësimin e vendit në BE, ndërkohë që si një hap i  rëndësishëm institucional mund të vlerësohet ngritja e Komitetit Shtetëror të  Minoriteteve,2 si institucion qëndror me natyrë konsultative, në varësi të  Kryeministrit.

Efektivisht sot, shoqëria shqiptare ka vendosur një balancë pozitive  përsa i përket çështjes dhe respektimit të të drejtave të pakicave, duke u  renditur ndër shtetet, të cilat janë angazhuar në përmbushje të standardeve  ndërkombëtare në këtë fushë. Zbatimi në praktikë i parashikimeve të akteve  themelore, por edhe zbatimi i akteve ndërkombëtare dhe veçanërisht i  përcaktimeve të Konventës Kuadër të Këshillit të Europës “Për mbrojtjen e  minoriteteve kombëtare”, ndikojnë në mënyrë të drejtpërdrejtë në përmirësimin  e të drejtave të pakicave.

Por pavarësisht parashikimeve ligjore kombëtare dhe ndërkombëtare që janë në fuqi, të gjithë jemi të vetëdijshëm, që ka shumë për të bërë, për të  garantuar në praktikë, respektimin e të drejtave, integrimin dhe përfshirjen  sociale, të pakicave që jetojnë në vendin tonë. Në këtë drejtim, roli i Avokatit  të Popullit merr një rëndësi të veçantë.

Në fakt, të drejtat e minoriteteve, nuk ishin të përfshira për disa kohë në  listën e të drejtave të njeriut, ndërkombëtarisht të njohura. Kjo situatë ka  ndryshuar materialisht vetëm gjatë dekadës së fundit, ndërkohë që  emancipimi për të drejtat e minoriteteve nuk mund te konsiderohet si i zgjidhur  përfundimisht madje edhe sot.
\end{document}
