\documentclass{article}[a4]
\begin{document}
\title{Fuqia e fjales}
\date{10, Shkurt 2021}
\author{Kristian Blido}
\maketitle

"Fjala vret më shumë se plumbi", shprehje tepër e njohur dhe pak herë e analizuar. A është fjala vërtet më vrastare se plumbi, a është fjala më e dhimbshme se vetë vdekja, a është jeta pa fjalë me keq se vdekja me fjalë, a ekziston fjala pas vdekjes?

Fjala është mjeti i komunikimit që evolucioni e ka nxjerrë si më te përshtatshmin dhe efektivin. Nuk e dimë kush ka qenë i pari qe "foli" dhe as çfare ka thënë, por me siguri ka qenë shumë mbreslënëse përderisa filluan ta bënin të gjithë. Fjala është manifestimi ndjesor i koshiencës së njeriut; unë mendoj, unë flas, ndonjëhere flas edhe pa menduar. Mendimi që gjykohet si i denjë për tu shpërndarë, flitet me fjalë dhe nëpërmjet ajërit transmetohet në një tjetër koshiencë, ku rigjykohet. Rigjykohet fjala, rigjykohet edhe vlera e atij që foli. Fjala është jo më e mirë se ai që e flet. Në vetvete, fjala, ështe vetëm një mjet dhe si e tillë është po aq e fuqishme, mirëbërëse ose keqbërëse sa ai që e përdor.

Fjala është pjesë e një sistemi më të madhë, të standardizuar, të quajtur gjuhë. Këto janë të shumta, por megjithëse të ndryshme në pamje dhe tingull, bazohen tek e njëjta strukturë: mendja e njeriut. Gjuhët si pasojë e përdorimit të shumtë, si në kohë ashtu edhe në numër, kanë evoluar përtej komunikimit të drejpërdrejtë. Gjuhët e kohës moderne kanë edhe nuanca estetike. Kjo vjen si pasojë e sufiçensës së fjalëve, shumë fjalë, e njëjta ide. Kjo ka sjellë që fjalë që i referohen të njëjtës ide, kanë ngjyrime emocionale të ndryshme. Thënë kjo, mund të themi qe fjalët kanë jo vetëm kuptim logjik, por edhe ndikim emocional. Mjet i ketillë, në duart e një profesionisti, bëhet plumb e kaluar plumbit.

Të përgjithësoje fuqinë e fjalës do ishte si te thoje që të gjithë afrikanët janë zeshkanë, shumica janë, por jo t'gjithë.
\end{document}
