\documentclass[12pt, a4paper]{article}

\usepackage[utf8]{inputenc}
\usepackage[T1]{fontenc}
\usepackage{textcomp}
\usepackage{amsmath, amssymb}

\title{Ese: Vetmia}
\author{Kristian Blido}
\date{24, Shkurt 2021}
\begin{document}
	\maketitle
Keni qene ndonjehere vetem? Keni qene ndonjehere vertet vetem? Keni qene ndonjehere aq vetem sa edhe vetja iu ka braktisur?

Shpesh fjala vetmi perdoret per te kumtuar largesine fizike dhe emocionale nga individe te tjere. Dhe ne kete kuptim eshte nder gjendjet qe krijon nje set emocionesh te zymta. I vetmuari ndihet i pasigurt, ndihet i lene menjane, ndihet i paafte, ndihet thuajse njeri. Duke i mohuar dikujt nderveprimin me njerez te tjeret, i ke mohuar qenien, ke mohur qenien tende. Njerezit jane qenie sociale, qe, megjithese goxha te afte ne shume gjera, e kane thelbesore jetesen ne tufe.

Megjithate, epiqendra e vetmise se vertete eshte ekzistenca. Me ekzistence nuk deshiroj te pershkruaj ekzistencen materiale, por ekzistencen e vetedijes, ekzistencen e arsyetimit, ekzistencen e mendjes. Kur te braktis edhe vetja jote nuk ka pikture e poezi, kenge e valle, njerez te njohur ose jo qe te largon rete e territ, boshllekun emocional dhe deshiren per te gjetur fundin. Gjendje e ketille, mbase mes njerezve, por pa veten eshte me e shpeshte midis personave qe vuajne nga semundje te ngjashme me demencen, por edhe nga njerez pa semundje kronike mendore. I ketille konglomerat ndjenjash eshte i paperballueshem per ca dhe vujtje e vazhdueshme per ca te tjere, por sduhet harruar qe i vetem apo jo, mes mureve apo mes stadiumit, ti je ti, sepse askush s'mund te jete ti.

Vetmia sociale duhet vleresuar disi me shume. Jane jo te paket njerezit qe jane gjithmone duke kerkuar barin me te ri, festen me me shume njerez ose numrin me te madh te mundshem te miqve. Kur je larg te tjereve, je me afer vetes. Vetmia sociale duhet pare si nje moment vete reflektimi, vetepermiresimi dhe vetepranimi. "Nuk kam qene asnjehere i vetmuar, une jam argetimi me i mire per mua." - Bukowski
\end{document}
